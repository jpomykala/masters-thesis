\chapter*{Streszczenie}
Celem niniejszej pracy dyplomowej było opracowanie systemu automatycznego przypisywania tekstów w języku polskim do grup tematycznych. Zapoznanie się z narzędziami NLP (ang. natural language processing) dla języka polskiego oraz porównanie efektywności algorytmów klasyfikacji z nauczycielem. W pracy wykorzystano dwa różne modele: \textit{Bag-Of-Words} wraz z normalizacją \textit{tf-idf} oraz \textit{fastText}, który jest rozwinięciem modelu \textit{word2vec}. Badania zostały przeprowadzone na dwóch korpusach danych: zebranych artykułach z popularnych stron internetowych oraz porównawczym udostępnionym w repozytorium Clarin.

Głównymi wyzwaniami w pracy było:

\begin{itemize}
\item Zebranie korpusu dokumentów w języku polskim z informacjami o przynależności do grup tematycznych,
\item opracowanie algorytmów przetwarzania tekstów z wykorzystaniem metod NLP dla języka polskiego,
\item implementacja programowa zaproponowanych algorytmów,
\item nauczenie i przetestowanie klasyfikatorów,
\item modyfikacja metod przetwarzania dokumentów.
\end{itemize}

W pracy wykorzystano następujące technologie:
\begin{itemize}
\item Java 9
\item Spring Boot
\item Python 3.6
\item SciKit Learn
\item fastText
\end{itemize}

